\chapter*{Abstract}
When working on any IoT (Internet of Things) project, security should always be a top priority, this because the more devices are connected to the internet, the more attractive the data becomes for cybercriminals.
All of these embedded systems deal with sensitive data that can be harmful if it falls into the wrong hands. Even the smallest Internet of Things system can reveal a lot about the real world in which these devices operate, and they may even be able to access and control it themselves.
As a result, may IoT projects have begun to implement security measures. Implementing a Virtual Private Network (VPN) is one of the most popular solutions.
Through this system users can send and receive data across public networks as if their computers were directly connected to a private network, knowing that all of their data is encrypted and their IP address is hidden.
For this purpose, this paper suggests to use WireGuard as a VPN and see if it is a viable security solution for an IoT project. 
WireGuard uses state-of-the-art cryptography, which makes it faster, more secure, and more friendly to mobile and IoT devices than other protocols like OpenVPN or IPsec.
To test this solution, an ESP32 was chosen as the development board, with FreeRTOS as the Real Time Operating System, and a TCP/IP communication was established.