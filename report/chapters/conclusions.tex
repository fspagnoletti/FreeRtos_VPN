\chapter{Conclusions}
Starting with the ESP32 board and then porting in simulation on Linux FreeRTOS, the goal of implementing a VPN Client for FreeRTOS was met.
Several issues arose during the project's development. The first step was to research what a Virtual Private Network is, as well as its functions and how to use it.
\\Following some evaluations, WireGuard was selected as the protocol to create an encrypted communication that ensures confidentiality, integrity, and authentication.
The ESP32 was chosen as the client's development board because it provides all of the required utilities at a low cost.
The Espressif's official IoT Development Framework (ESP-IDF) functionalities were studied in order to use them on the board.
\\While browsing the internet, examples of previously created WireGuard implementations on IoT devices were discovered and used as a starting point for the project.
To simulate a simple packet transmission, a TCP/IP port on a Linux machine was opened, and the ESP32 was used as the sender.
Following the successful communication, the WireGuard module was added to ensure a secure channel.
To ensure that the packets were encrypted, an Internet traffic analyser was used.
So, after the previously described implementation, it was easier to understand how the WireGuard module works and how to configure it. This enabled it to be ported in a more general scenario using the FreeRTOS simulator for Linux.
\\All of the work done allowed the establishment of a raw TCP/IP communication, but a more complex application exploiting the full potential of a VPN could be developed.
Other improvements must be made before this solution can be used in a real-world scenario.
